\documentclass{beamer}
\usepackage[utf8]{inputenc}
\usepackage{graphicx}
\usepackage{booktabs}
\usepackage{amsmath}
\setbeamertemplate{footline}{}
\setbeamercolor*{normal text}{fg=white!70,bg=black!80}
\setbeamercolor*{structure}{fg=black}
\setbeamercolor{title}{fg=white}
\setbeamercolor*{alerted text}{fg=red!85!black}
\setbeamercolor*{item projected}{use=item,fg=black,bg=item.fg!35}
\setbeamercolor*{palette primary}{use=structure,fg=structure.fg}
\setbeamercolor*{palette secondary}{use=structure,fg=structure.fg!95!black}
\setbeamercolor*{palette tertiary}{use=structure,fg=structure.fg!90!black}
\setbeamercolor*{palette quaternary}{use=structure,fg=structure.fg!95!black,bg=black!80}
\setbeamercolor*{framesubtitle}{fg=white}

\setbeamercolor*{block title}{parent=structure,bg=black!60}
\setbeamercolor*{block body}{fg=white,bg=black!10}
\setbeamercolor*{block title alerted}{parent=alerted text,bg=black!15}
\setbeamercolor*{block title example}{parent=example text,bg=black!15}
\usefonttheme{structuresmallcapsserif}
\title{Face Analysis \& Recommendation System via Reinforcement Learning}
\subtitle{Contextual Thompson Sampling for Personalized Celebrity Look-alike}
\author{Ivan Chabanov \and Aleksandr Michailov \and Nikita Shiyanov}
\institute{RL Course Project}
\date{\today}
\usetheme{Madrid}
\begin{document}
\setbeamercolor{background canvas}{bg=black!80}
\frame{\titlepage}
% -------------------------------------------------------------------------
% SECTION 1: Introduction & Problem Recap (1-2 minutes)
% -------------------------------------------------------------------------
\section{Introduction \& Problem Definition}
\begin{frame}{Problem Context}
    \begin{itemize}
        \item \textbf{Subjectivity of Beauty:} Facial attractiveness and celebrity resemblance are highly subjective metrics.
        \item \textbf{Static Model Limitations:} Traditional classifiers provide a fixed output (e.g., "You look like Brad Pitt") which may not align with every user's perception.
        \item \textbf{The Need for Adaptation:} A system is required that learns from individual user feedback to refine its recommendations over time.
    \end{itemize}
\end{frame}
\begin{frame}{Project Objectives}
    \begin{block}{Core Goal}
        Develop a mobile application that uses Computer Vision for feature extraction and Reinforcement Learning for personalized recommendations.
    \end{block}
    \begin{itemize}
        \item \textbf{Facial Analysis:} Accurately estimate attractiveness scores and identify celebrity look-alikes.
        \item \textbf{Personalization:} Implement an RL agent that adapts to user preferences.
        \item \textbf{Continuous Learning:} Enable the system to improve post-deployment via online updates.
    \end{itemize}
\end{frame}
% -------------------------------------------------------------------------
% SECTION 2: Methodology & RL Approach (4-5 minutes)
% -------------------------------------------------------------------------
\section{Methodology: Reinforcement Learning}
\begin{frame}{Why Reinforcement Learning?}
    We formulated the problem as a \textbf{Contextual Multi-Armed Bandit}.
    
    \begin{table}[]
        \centering
        \begin{tabular}{l|l}
            \toprule
            \textbf{Requirement} & \textbf{RL Solution (Thompson Sampling)} \\
            \midrule
            User preferences differ & Adapts to individuals vs. global model \\
            Need exploration & Actively explores uncertain candidates \\
            Improvement over time & Learns online after deployment \\
            Uncertainty-aware & Beta distribution tracks confidence \\
            \bottomrule
        \end{tabular}
    \end{table}
\end{frame}
\begin{frame}{Formal Model Definition}
    \begin{itemize}
        \item \textbf{State (Context):} 
        \begin{itemize}
            \item 512-dimensional facial embedding from \textbf{Inception ResNet V1}.
            \item Encodes geometry, landmarks, and texture.
        \end{itemize}
        
        \item \textbf{Actions (Arms):}
        \begin{itemize}
            \item Each celebrity in the dataset represents an "arm".
            \item Action: Select Top-K celebrities to present.
        \end{itemize}
        
        \item \textbf{Reward Signal:}
        \begin{itemize}
            \item Binary user feedback (Like / Dislike).
            \item \textbf{Update Rule:} Like $\rightarrow \alpha \leftarrow \alpha + 1$; Dislike $\rightarrow \beta \leftarrow \beta + 1$.
        \end{itemize}
    \end{itemize}
\end{frame}
\begin{frame}{Thompson Sampling Mechanism}
    We maintain a \textbf{Beta($\alpha_i, \beta_i$)} distribution for each celebrity $i$.
    
    \begin{block}{Decision Loop}
    \begin{enumerate}
        \item \textbf{Sample:} Draw probability $p_i \sim \text{Beta}(\alpha_i, \beta_i)$ for each celebrity.
        \item \textbf{Contextualize:} Adjust score using cosine similarity between user embedding and celebrity prototype.
        \item \textbf{Rank:} Select Top-K candidates with highest adjusted scores.
        \item \textbf{Update:} Receive user feedback and update posterior parameters $\alpha, \beta$.
    \end{enumerate}
    \end{block}
    
    \textit{"This approach naturally balances exploration of new candidates with exploitation of known successful matches."}
\end{frame}
% -------------------------------------------------------------------------
% SECTION 3: System Implementation (4-5 minutes)
% -------------------------------------------------------------------------
\section{System Implementation}
\begin{frame}{Technical Architecture}
    \begin{columns}
        \column{0.5\textwidth}
        \textbf{Backend}
        \begin{itemize}
            \item \textbf{Framework:} FastAPI (Python).
            \item \textbf{ML Core:} PyTorch.
            \item \textbf{Database:} PostgreSQL (User data and RL states).
            \item \textbf{Infrastructure:} Docker containerization.
            \item \textbf{Tracking and Metrics:} MLFlow, Grafana, Prometheus, OpenTelemetry, Loki, Tempo
            \item \textbf{Data versioning:} DVC
        \end{itemize}
        
        \column{0.5\textwidth}
        \textbf{Frontend}
        \begin{itemize}
            \item \textbf{Framework:} Flutter (Mobile).
            \item \textbf{State Management:} Bloc/Cubit.
            \item \textbf{Camera:} Native device integration.
            \item \textbf{Design:} iOS (Cupertino) Design.
        \end{itemize}
    \end{columns}
\end{frame}
\begin{frame}{Technical Architecture}
    \begin{figure}
        \centering
        \includegraphics[width=\textwidth]{arch.png}
    \end{figure}
\end{frame}
\begin{frame}{Technical Architecture}
    \begin{figure}
        \centering
        \includegraphics[width=\textwidth]{rl.png}
    \end{figure}
\end{frame}
\begin{frame}{ML Models and Pipeline}
    \begin{enumerate}
        \item \textbf{Attractiveness Classifier:}
        \begin{itemize}
            \item \textbf{Architecture:} ResNet-50.
            \item \textbf{Dataset:} SCUT-FBP5500 (5,500 frontal faces).
            \item \textbf{Task:} Regression (Score 0.0 - 1.0).
        \end{itemize}
        
        \item \textbf{Celebrity Matcher (Feature Extractor):}
        \begin{itemize}
            \item \textbf{Architecture:} Inception Resnet V1 (pretrained on VGGFace2).
            \item \textbf{Dataset:} Open Famous People Faces (258 classes).
            \item \textbf{Task:} Face recognition \& embedding generation.
        \end{itemize}
    \end{enumerate}
\end{frame}
\begin{frame}{Team Contributions}
    \begin{itemize}
        \item \textbf{Aleksandr Mikhailov:} 
        \begin{itemize}
            \item Backend architecture design.
            \item Database schema design.
            \item ML Models design, training
            \item Dataset collection
            \item Deployment
        \end{itemize}
        
        \item \textbf{Nikita Shiyanov:}
        \begin{itemize}
            \item RL Algorithm implementation (Thompson Sampling).
        \end{itemize}
        
        \item \textbf{Ivan Chabanov:}
        \begin{itemize}
            \item Frontend development (Flutter).
            \item UI/UX design and camera integration.
            \item API integration and state management.
        \end{itemize}
    \end{itemize}
\end{frame}
% -------------------------------------------------------------------------
% SECTION 4: Results & Evaluation (3-4 minutes)
% -------------------------------------------------------------------------
\section{Results & Evaluation}
\begin{frame}{Model Performance}
    \textbf{Attractiveness Classifier (ResNet-50):}
    \begin{itemize}
        \item Achieved low Mean Squared Error (MSE) on validation set.
        \item High Pearson correlation with human ratings.
    \end{itemize}
    
    \vspace{0.5cm}
    
    \textbf{Celebrity Matcher (Inception Resnet V1):}
    \begin{itemize}
        \item High Top-1 and Top-5 Accuracy on Open Famous People Faces dataset.
        \item Robust embedding generation for diverse lighting conditions.
    \end{itemize}
\end{frame}
\begin{frame}{RL Agent Analysis}
    We track the following metrics to validate the online learning process:
    
    \begin{itemize}
        \item \textbf{Cumulative Reward:} Global "happiness" score increases over time as the agent learns.
        \item \textbf{Exploration Rate:} Decreases naturally as $\alpha$ and $\beta$ counts grow, but never reaches zero (always open to new trends).
        \item \textbf{Regret:} The difference between optimal choice and agent choice diminishes, proving convergence.
    \end{itemize}
    
    \begin{block}{Insight}
        Unlike static systems, our agent identifies "polarizing" celebrities (high variance in Beta distribution) vs. "universally liked" ones.
    \end{block}
\end{frame}
% -------------------------------------------------------------------------
% SECTION 5: Conclusion (1 minute)
% -------------------------------------------------------------------------
\section{Conclusion}
\begin{frame}{Summary and Future Work}
    \textbf{Conclusion:}
    \begin{itemize}
        \item Successfully integrated State-of-the-Art CV models with Contextual Bandits.
        \item The system moves beyond static prediction to \textbf{adaptive recommendation}.
        \item "This turns our model into a live adaptive system, learning from humans in real time."
    \end{itemize}
    
    \vspace{0.5cm}
    
    \textbf{Future Work:}
    \begin{itemize}
        \item \textbf{Cold Start:} Incorporate meta-learning for new users.
        \item \textbf{A/B Testing:} Deploy different priors to optimize convergence speed.
    \end{itemize}
\end{frame}
\end{document}